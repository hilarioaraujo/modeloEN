% Capítulo 1
% 
\chapter{Estrutura da Dissertação de Mestrado} % Título do capítulo
\label{chap:Chapter1} % Para fazer referência a esta secção ao longo da dissertação, use o comando Chapter~\ref{Chapter1}


%-------------------------------------------------------------------------------
%---------
%
\section{Introdução} 
\label{sec:chap1_introduction} %Para fazer referência a esta secção ao longo da dissertação, use o comando Section~\ref{sec:chap1_introduction}

A introdução prepara o leitor para uma leitura organizada e uma melhor compreensão do trabalho científico que se está a apresentar. Deve por isso começar por apresentar, de forma breve, a problemática em que se insere o trabalho científico.
Sempre que for necessário fazer este enquadramento de forma detalhada e mais longa, deve apenas referir-se o assunto na introdução indicando que a apresentação mais detalhada será feita num dos primeiros capítulos (normalmente o primeiro).

Feita, contudo, esta apresentação do assunto, expondo a problemática subjacente ao mesmo, deve apresentar-se o problema, ou problemas, objeto do estudo efetuado, a que se segue uma explicação das vias seguidas na investigação e da forma como elas transparecem na estrutura adotada para a apresentação. Seguem-se, se forem relevantes e necessárias, algumas explicações sobre a metodologia do trabalho, terminando com os objetivos que se pretende alcançar. Sempre que for necessário, para uma melhor compreensão por parte do leitor, pode dividir-se a introdução em pequenos subcapítulos, indicando objetivos, vias seguidas na investigação, metodologias e resultados pretendidos.

Este capítulo não tem numeração específica e deve iniciar-se ao cimo de uma página ímpar (à direita), independentemente do facto da página anterior ser deixada com verso em branco. 



\section{Capítulos}

Segue-se o corpo principal do trabalho, dividido em capítulos, numerados em numeração árabe (1, 2, 3,...), que podem subdividir-se em subcapítulos, sucessivamente, igualmente numerados segundo a lógica

1. Capítulo 

1.1 Subcapítulo 

1.1.1 Sub-subcapítulo 

etc...

Aos capítulos e subcapítulos devem ser dados títulos, em letra destacada em negrito, de corpo sucessivamente 14, 13 e 12, sempre encostados à margem esquerda da página sem qualquer avanço.

Não é possível apresentar um critério único para o ordenamento de capítulos e subcapítulos, decorrendo esta estrutura da natureza do próprio trabalho, variando consoante a área disciplinar ou científica do mesmo e das suas características próprias.\\
Nalguns casos terá uma natureza explicativa, noutros passará pela exposição de resultados e sua interpretação, envolvendo a apresentação de critérios, tabelas de resultados, memória descritiva, etc.

Cada um dos capítulos deve começar ao cimo de uma página ímpar (à direita).

\section{Conclusão}
A conclusão segue-se ao corpo principal dos capítulos que constituem o trabalho, realçando, de forma resumida e nos aspetos mais relevantes, os passos seguidos e os resultados obtidos (mas evitando fazer um resumo que repita aspetos do corpo). Devem expor-se as dificuldades e limitações sentidas, sobretudo se as mesmas limitaram a investigação e prejudicaram o alcançar dos resultados propostos na introdução. E, de igual modo, se a investigação desenvolvida mostrou novas vias de trabalho que não puderam ser desenvolvidas, devem evidenciar-se os caminhos que foram abertos, avançando com sugestões e propostas para trabalhos futuros que deem continuidade ao projeto presente.

Este capítulo não deve ser numerado, devendo começar ao início de uma página Ímpar (à direita), mesmo que a página anterior se encontre em branco.