%Modelo da Dissertação de Mestrado da EN
% TMDEI Thesis EN Style
% 
% Based on MastersDoctoralThesis Version 1.2 by Vel (vel@latextemplates.com) and
% Johannes Böttcher, downloaded from (21/11/15):
% http://www.LaTeXTemplates.com
%
% Authors:
%  Vel (vel@ latextemplates. com)
%  Johannes Böttcher%
%
% Adapted to Thesis EN Style (JUL/2019) by 
%  HILÁRIO ARAÚJO (rocha.araujo@marinha.pt) 
%  Ricardo Moura (ricardo.pinto.moura@marinha.pt)
%
%%%%%%%%%%%%%%%%%%%%%%%%%%%%%%%%%%%%%%%%%

%-----------------------------------------%
%    CONFIGURAÇÃO INICIAl
%-----------------------------------------%

\documentclass[12pt, % O tamanho da fonte do documento padrão, opções: 10pt, 11pt, 12pt
%oneside, % Dois lados (margens alternadas) para ligação por padrão, uncomment a mudar para um lado (para fins de desenho / leitura)
portuguese, %portuguese,% for Portuguese; english for English; delete temporary files if you change language (e.g. 'make clean; make')
singlespacing, % Espaçamento de linha única, alternativas: onehalfspacing or doublespacing (para fins de redação / leitura)
%draft, % Uncomment para ativar o modo de rascunho (sem imagens, sem links, caixas de texto overfull indicadas)
%nolistspacing, % Se o documento for de meio-espaço ou de espaçamento duplo, uncomment para definir o espaçamento nas listas como único
%liststotoc, % Uncomment para adicionar a lista de figuras / tabelas / etc ao índice (não recomendado)
%toctotoc, % Uncomment para adicionar o índice principal ao índice (não recomendado)
parskip, % Adicione espaço entre parágrafos (recomendado)
%nohyperref, % Uncomment para não carregar o pacote hyperref (não recomendado)
nohyperreflinkcolor, % links de hyperref não são coloridos (comment para links de cores, por exemplo, para produzir uma versão somente eletrônica)
headsepline, % Uncomment para obter uma linha sob o cabeçalho
]{enStyle} % O arquivo de classe que especifica a estrutura do documento

\usepackage{tikz} % Required for creating graphics programmatically (can be removed if not used)
%\usetikzlibrary{arrows} % Required for fancy arrows in TiKZ graphics (can be removed if not used)

\usepackage{pgfplots} % Required for drawing high--quality function plots (can be removed if not used)
\pgfplotsset{compat=newest}

%
% Next you have examples of admissable citation styles; we recomend using the authoryear-comp citation style (which resembles Harvard); don't forget to only uncomment one
%

% authoryear-comp: recommended citation style (e.g. (Buendía, 1860), (Buendía 1910, Arcadio 1940))
\usepackage[style=authoryear-comp,backend=biber]{biblatex} % Bibtex backend with the authoryear-comp citation style (authoryear citations, bibliography ordered alphabetically)

% numeric citation style (e.g. [1], [1-3])
%\usepackage[style=numeric-comp,sorting=none,backend=biber]{biblatex} % Bibtex backend with the numeric-comp citation style (numeric citations, bibliography ordered by appearance)

% alphabetic citation style (e.g. [Buendía10], [Buendía10, Arcadio40])
%\usepackage[style=alphabetic,sorting=none,backend=biber]{biblatex} % Bibtex backend with the alphabetic citation style (alphabetic citations, bibliography ordered by appearance)


\addbibresource{mainbibliography.bib} % O nome da base de dados da sua bibliografia

\makeglossaries % cria o glossário


%----------------------------------------------------------------------------------------
%	INFORMAÇÃO SOBRE A TESE
%----------------------------------------------------------------------------------------

\thesistitle{Título da Dissertação de Mestrado} % Escreva o título da tese, pode referenciar o titulo ao longa da dissertação utilizando o comando (\ttitle)

\thesissubtitle{Subtítulo da Dissertação de Mestrado} % %Escreva o subtítulo da tese, pode referenciar o titulo ao longa da dissertação utilizando o comando (\subttitle)

\author{Nome do \textsc{Orientando}} % % Escreva o seu nome, pode referenciar o autor ao longa da dissertação utilizando o comando (\authorname)

\subjectarea{Engenharia Naval Ramo de Armas e Eletrónica} % Identifica a tua especialidade: Engenharia Naval Ramo de Armas e Eletrónica, Engenharia Naval Ramo de Mecânica, Marinha, Fuzileiros, Marinha e Administração Naval
%poderá referenciar-lo ao longo do seu trabalho com o comando (\areaname)

\supervisor{Nome do \textsc{Orientador}} % O nome do seu supervisor, isto é usado na página da contra capa,  poderá referenciar-lo ao longo do seu trabalho com o comando (\supname)

%\cosupervisor{Dr. Jack \textsc{Smith}} % O nome do seu co-orientador, isto é usado na página da contra capa,  poderá referenciar-lo ao longo do seu trabalho com o comando (\cosupname) (comenta, se não tiveres co-orientador)

\keywords{Palavra-chave 1, ..., Palavra-chave 2} % Defina até 6 palavras-chave que descrevam melhor o seu trabalho, e poderá referenciar-las ao longo do seu trabalho com o comando (\keywordnames)

\university{\href{https://escolanaval.marinha.pt}{Escola Naval}} % Nome da universidade e endereço web, , e poderá referenciar-la ao longo do seu trabalho com o comando \univname

\department{\href{http://department.university.com}{Departamento Ciência e Tecnologias}} % Nome do departamento e endereço web, e poderá referenciar-la ao longo do seu trabalho com o comando \deptname
% Classes e respetivos departamentos:
% EN-AEL E EN-MEC - Departamento Ciência e Tecnologias
% AN - Departamento de Humanidades e Gestão
% M - Departamento de Ciências do Mar
% FZ - Departamento de Ciências do Mar

\thesisdate{Escola Naval, \today} % data da impressão da tese, pode referenciar a data ao longa da dissertação utilizando o comando (\tdate)

\hypersetup{pdftitle=\ttitle} % Set the PDF's title to your title
\hypersetup{pdfauthor=\authorname} % Set the PDF's author to your name
\hypersetup{pdfkeywords=\keywordnames} % Set the PDF's keywords to your keywords

\begin{document}

%----------------------------------------------------------------------------------------
%	PÁGINAS INICIAIS
%----------------------------------------------------------------------------------------

% Inclui as páginas iniciais da sua tese
%-----------------------------------------%
% O frontmmatter são as chamadas páginas iniciais, terá de atualizar as respetivas secções.
%-----------------------------------------%

%-----------------------------------------%
%--------CONFIGURAÇÃO INICIAL-------------%
%-----------------------------------------%

%Todos os acrônimos devem ser escritos neste arquivo.

\newacronym{RTS}{RTS}{Real-Time System}
\newacronym{GPOS}{GPOS}{General Purpose Operating System}
\newacronym{RTOS}{RTOS}{Real-Time Operating System}
\newacronym{PGF}{PGF}{Portable Graphics Format}


\frontmatter % Use roman page numbering style (i, ii, iii, iv...) for the pre-content pages

\pagestyle{plain} % Default to the plain heading style until the thesis style is called for the body content


%-----------------------------------------%
%-----------CAPA e CONTRA CAPA-------------------------%
%-----------------------------------------%

\makethetitlepage


\makeconttitlepage



%-----------------------------------------%
%-----------EPÍGRAFE-------------------------%
%-----------------------------------------%
\begin{epigraph}
\null \vfill
\begin{flushright}

A epígrafe traduz-se pela inscrição de sentença conceituosa que, de algum modo inspirou o autor na elaboração do trabalho ou nas suas ações correntes, e que o mesmo considere importante revelar no trabalho. Tem uma natureza facultativa.

\end{flushright}
\vfill \null%
\end{epigraph}%


%-----------------------------------------%
%-----------DEDICATÓRIA-------------------------%
%-----------------------------------------%
\begin{dedicatory}
\null \vfill
\begin{flushright}


A dedicatória tem por finalidade prestar homenagem ou dedicar o trabalho a alguém próximo ou que tenha um especial significado para o autor do trabalho. É, também, um elemento facultativo na estrutura do trabalho, mas é usual que seja feita dedicando o trabalho aos pais, à família mais chegada ou a alguém com relevância especial na vida do autor. 


\end{flushright}
\vfill \null%
\end{dedicatory}%


%-----------------------------------------%
%-----------AGRADECIMENTOS-------------------------%
%-----------------------------------------%
\begin{acknowledgements}

Agradecimento é a expressão registada de uma gratidão às pessoas, entidades ou instituições que, de algum modo, contribuíram para a elaboração do trabalho. Sendo um elemento opcional, quando exista deve incluir-se na frente de folha a colocar logo após a folha de rosto ou das folhas da epígrafe e/ou da dedicatória, deixando o verso em branco.

\end{acknowledgements}%



%-----------------------------------------%
%-----------RESUMO-------------------------%
%-----------------------------------------%
\begin{abstract}

\noindent \textbf{\ Subtítulo caso queira!}

\noindent Segue-se, com caráter obrigatório, um resumo em língua portuguesa e em língua inglesa (abstract), cada um deles com um máximo de 300 palavras.
Após cada um dos resumos devem ser indicadas cinco palavras-chave – português e inglês – para indexação futura.


% As palavras chave terão de ser definidas no ficheiro main.tex depois da linha de código keywords
\end{abstract}


%-----------------------------------------%
%-----------ABSTRACT-------------------------%
%-----------------------------------------%

\begin{abstractotherlanguage}

\noindent \textbf{\ Subtitle if you want}

\noindent Write here the abstract.

\noindent \newline Key words:

\end{abstractotherlanguage}


%-----------------------------------------%
%--ÍNDICE DE CONTEÚDO / FIGURAS / TABELAS-%
%-----------------------------------------%

\tableofcontents % Imprime o índice principal

\listoffigures % Imprime a lista de figuras

\listoftables % Imprime a lista de tabelas

\iflanguage{portuguese}{
\renewcommand{\listalgorithmname}{Lista de Algor\'itmos}
}
\listofalgorithms % Imprime a lista de algoritmos
%\addchaptertocentry{\listalgorithmname} %Uncomment para mostrar no índice a lista de algoritmos


\renewcommand{\lstlistlistingname}{List of Source Code}
\iflanguage{portuguese}{
\renewcommand{\lstlistlistingname}{Lista de C\'odigo}
}
\lstlistoflistings % Imprime a lista de listagens (código-fonte da linguagem de programação)

%\addchaptertocentry{\lstlistlistingname} %Uncomment para mostrar a lista de de codigo no indice


%-----------------------------------------%
%------------ABREVIATURAS-----------------%
%-----------------------------------------%
%\begin{abbreviations}{ll} % IncluI uma lista de abreviações (uma tabela de duas colunas)

%%\textbf{LAH} & \textbf{L}ist \textbf{A}bbreviations \textbf{H}ere\\
%%\textbf{WSF} & \textbf{W}hat (it) \textbf{S}tands \textbf{F}or\\
%\end{abbreviations}

%-----------------------------------------%
%------------SÍMBOLOS-----------------%
%-----------------------------------------%

\begin{symbols}{lll} % IncluI uma lista de símbolos (uma tabela de três colunas)

$a$ & distance & \si{\meter} \\
$P$ & power & \si{\watt} (\si{\joule\per\second}) \\
%Símbolo, nome e unidade \\

\addlinespace % espaçamento para separar os símbolos romanos dos grego

$\omega$ & angular frequency & \si{\radian} \\

\end{symbols}



%-----------------------------------------%
%-----------ACRÓNIMOS----------------%
%-----------------------------------------%

\newcommand{\listacronymname}{List of Acronyms}
\iflanguage{portuguese}{
\renewcommand{\listacronymname}{Lista de Acr\'onimos}
}

%Use GLS
\glsresetall
\printglossary[title=\listacronymname,type=\acronymtype,style=long]

%----------------------------------------------------------------------------------------
%	DONE
%----------------------------------------------------------------------------------------

\mainmatter % Começar numeração da página com numéros árabes (1,2,3 ...)
\pagestyle{thesis} % Colocar os cabeçalhos nas páginas com o estilo defenido para o corpo da tese

%----------------------------------------------------------------------------------------
%	CORPO DA TESE
%----------------------------------------------------------------------------------------

% Inclui os capítulos da tese como das respetivas pastas separadas por cada capítulo
% Uncomment os capítulos à medida que vais criando novos.

% Capítulo 1
% 
\chapter{Estrutura da Dissertação de Mestrado} % Título do capítulo
\label{chap:Chapter1} % Para fazer referência a esta secção ao longo da dissertação, use o comando Chapter~\ref{Chapter1}


%-------------------------------------------------------------------------------
%---------
%
\section{Introdução} 
\label{sec:chap1_introduction} %Para fazer referência a esta secção ao longo da dissertação, use o comando Section~\ref{sec:chap1_introduction}

A introdução prepara o leitor para uma leitura organizada e uma melhor compreensão do trabalho científico que se está a apresentar. Deve por isso começar por apresentar, de forma breve, a problemática em que se insere o trabalho científico.
Sempre que for necessário fazer este enquadramento de forma detalhada e mais longa, deve apenas referir-se o assunto na introdução indicando que a apresentação mais detalhada será feita num dos primeiros capítulos (normalmente o primeiro).

Feita, contudo, esta apresentação do assunto, expondo a problemática subjacente ao mesmo, deve apresentar-se o problema, ou problemas, objeto do estudo efetuado, a que se segue uma explicação das vias seguidas na investigação e da forma como elas transparecem na estrutura adotada para a apresentação. Seguem-se, se forem relevantes e necessárias, algumas explicações sobre a metodologia do trabalho, terminando com os objetivos que se pretende alcançar. Sempre que for necessário, para uma melhor compreensão por parte do leitor, pode dividir-se a introdução em pequenos subcapítulos, indicando objetivos, vias seguidas na investigação, metodologias e resultados pretendidos.

Este capítulo não tem numeração específica e deve iniciar-se ao cimo de uma página ímpar (à direita), independentemente do facto da página anterior ser deixada com verso em branco. 



\section{Capítulos}

Segue-se o corpo principal do trabalho, dividido em capítulos, numerados em numeração árabe (1, 2, 3,...), que podem subdividir-se em subcapítulos, sucessivamente, igualmente numerados segundo a lógica

1. Capítulo 

1.1 Subcapítulo 

1.1.1 Sub-subcapítulo 

etc...

Aos capítulos e subcapítulos devem ser dados títulos, em letra destacada em negrito, de corpo sucessivamente 14, 13 e 12, sempre encostados à margem esquerda da página sem qualquer avanço.

Não é possível apresentar um critério único para o ordenamento de capítulos e subcapítulos, decorrendo esta estrutura da natureza do próprio trabalho, variando consoante a área disciplinar ou científica do mesmo e das suas características próprias.\\
Nalguns casos terá uma natureza explicativa, noutros passará pela exposição de resultados e sua interpretação, envolvendo a apresentação de critérios, tabelas de resultados, memória descritiva, etc.

Cada um dos capítulos deve começar ao cimo de uma página ímpar (à direita).

\section{Conclusão}
A conclusão segue-se ao corpo principal dos capítulos que constituem o trabalho, realçando, de forma resumida e nos aspetos mais relevantes, os passos seguidos e os resultados obtidos (mas evitando fazer um resumo que repita aspetos do corpo). Devem expor-se as dificuldades e limitações sentidas, sobretudo se as mesmas limitaram a investigação e prejudicaram o alcançar dos resultados propostos na introdução. E, de igual modo, se a investigação desenvolvida mostrou novas vias de trabalho que não puderam ser desenvolvidas, devem evidenciar-se os caminhos que foram abertos, avançando com sugestões e propostas para trabalhos futuros que deem continuidade ao projeto presente.

Este capítulo não deve ser numerado, devendo começar ao início de uma página Ímpar (à direita), mesmo que a página anterior se encontre em branco.
\input{ch2/chapter2}
\input{ch3/chapter3}
%\input{ch4/chapter4}
%\input{ch5/chapter5}


%----------------------------------------------------------------------------------------
%	BIBLIOGRAFIA
%----------------------------------------------------------------------------------------

\printbibliography[heading=bibintoc]


%----------------------------------------------------------------------------------------
%	APÊNDICES e ANEXOS
%----------------------------------------------------------------------------------------

% Incluir os apêndices da tese como arquivos separados da pasta (appendices)
% Uncomment as linhas para teres à medida que fores criando apendices.

\appendixfile{appendix1}%
%\appendixfile{appendix2}%
%\appendixfile{appendix3}%
%\appendixfile{appendix}%


% Incluir os anexos da tese como arquivos separados da pasta (annexes)
% Uncomment as linhas para teres à medida que fores criando anexos

\annexfile{annex1}%
%\annexfile{annex2}%
%\annexfile{annex3}%


%Imprime no documento os anexos e apendices.
\printappendixes
\printannexes

%----------------------------------------------------------------------------------------

\end{document}
