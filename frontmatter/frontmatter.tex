%O frontmmatter são as chamadas páginas iniciais, terá de atualizar as respetivas secções.

%-----------------------------------------%
%    CONFIGURAÇÃO INICIAL                 %
%-----------------------------------------%

%Todos os acrônimos devem ser escritos neste arquivo.

\newacronym{RTS}{RTS}{Real-Time System}
\newacronym{GPOS}{GPOS}{General Purpose Operating System}
\newacronym{RTOS}{RTOS}{Real-Time Operating System}
\newacronym{PGF}{PGF}{Portable Graphics Format}


\frontmatter % Use roman page numbering style (i, ii, iii, iv...) for the pre-content pages

\pagestyle{plain} % Default to the plain heading style until the thesis style is called for the body content

%----------------------------------------------------------------------------------------
%	CAPA
%----------------------------------------------------------------------------------------
\maketitlepage


%----------------------------------------------------------------------------------------
%	CONTRA CAPA
%----------------------------------------------------------------------------------------
\makeconttitlepage


%-----------------------------------------%
%      EPÍGRAFE     (opcional)                     
%-----------------------------------------%
\begin{epigraph}
\null \vfill
\begin{flushright}

A epígrafe traduz-se pela inscrição de sentença conceituosa que, de algum modo inspirou o autor na elaboração do trabalho ou nas suas ações correntes, e que o mesmo considere importante revelar no trabalho. Tem uma natureza facultativa.

\end{flushright}
\vfill \null%
\end{epigraph}%



%----------------------------------------------------------------------------------------
%	DEDICATÓRIA  (opcional)
%----------------------------------------------------------------------------------------
%\dedicatory{For/Dedicated to/To my\ldots}
\begin{dedicatory}
\null \vfill
\begin{flushright}

A dedicatória tem por finalidade prestar homenagem ou dedicar o trabalho a alguém próximo ou que tenha um especial significado para o autor do trabalho. 

É, também, um elemento facultativo na estrutura do trabalho, mas é usual que seja feita dedicando o trabalho aos pais, à família mais chegada ou a alguém com relevância especial na vida do autor. 

\end{flushright}
\vfill \null%
\end{dedicatory}


%----------------------------------------------------------------------------------------
%	AGRADECIMENTOS (opcional)
%----------------------------------------------------------------------------------------

\begin{acknowledgements}

Agradecimento é a expressão registada de uma gratidão às pessoas, entidades ou instituições que, de algum modo, contribuíram para a elaboração do trabalho. Sendo um elemento opcional, quando exista deve incluir-se na frente de folha a colocar logo após a folha de rosto ou das folhas da epígrafe e/ou da dedicatória, deixando o verso em branco.

\end{acknowledgements}




%----------------------------------------------------------------------------------------
%	RESUMO
%----------------------------------------------------------------------------------------

\begin{abstract}

\noindent \textbf{\ Subtítulo caso queira!}

Segue-se, com caráter obrigatório, um resumo em língua portuguesa e em língua inglesa (abstract), cada um deles com um máximo de 300 palavras.

Após cada um dos resumos devem ser indicadas cinco palavras-chave – português e inglês – para indexação futura.

% As palavras chave terão de ser definidas no ficheiro main.tex depois da linha de código keywords
\end{abstract}


%----------------------------------------------------------------------------------------
%	ABSTRACT 
%----------------------------------------------------------------------------------------
\begin{abstractotherlanguage}
% here you put the abstract in the "other language": English, if the work is written in Portuguese; Portuguese, if the work is written in English.


\noindent \textbf{\ Subtitle if you want}
\noindent 

Trabalhos escritos em língua Inglesa devem incluir um resumo alargado com cerca de 1000 palavras, ou duas páginas.

Se o trabalho estivesse escrito em Português, este resumo seria em língua Inglesa, com cerca de 200 palavras, ou uma página.

Para alterar a língua basta ir às configurações do documento no ficheiro \file{main.tex} e alterar para a língua desejada ('english' ou 'portuguese')\footnote{Alterar a língua requer apagar alguns ficheiros temporários; O target \keyword{clean} do \keyword{Makefile} incluído pode ser utilizado para este propósito.}. Isto fará com que os cabeçalhos incluídos no template sejam traduzidos para a respetiva língua.

% As palavras chave terão de ser definidas no ficheiro main.tex depois da linha de código conkeywords

\end{abstractotherlanguage}



%----------------------------------------------------------------------------------------
%	ÍNDICE DE CONTEÚDO / FIGURAS / TABELAS
%----------------------------------------------------------------------------------------

\tableofcontents % Imprime o índice principal
\pdfbookmark[0]{\contentsname}{toc}% Adiciona o índice aos bookmarks do pdf

\listoffigures % Imprime a lista de figuras

\listoftables % Imprime a lista de tabelas

\listofalgorithms % Prints the list of algorithms
%\addchaptertocentry{\listalgorithmname} %Uncomment para mostrar no índice a lista de algoritmos

\lstlistoflistings % Imprime a lista de listagens (código-fonte da linguagem de programação)

%\addchaptertocentry{\lstlistlistingname} %Uncomment para mostrar a lista de de código no índice


%----------------------------------------------------------------------------------------
%	ABREVIATURAS
%----------------------------------------------------------------------------------------

\begin{abbreviations}{ll} % IncluI uma lista de abreviações (uma tabela de duas colunas)

%List of Abreviations
%
\textbf{LAH} & \textbf{L}ist \textbf{A}bbreviations \textbf{H}ere\\
\textbf{WSF} & \textbf{W}hat (it) \textbf{S}tands \textbf{F}or\\

\end{abbreviations}

%----------------------------------------------------------------------------------------
%	SÍMBOLOS
%----------------------------------------------------------------------------------------

\begin{symbols}{lll} % Inclui uma lista de símbolos (uma tabela de três colunas)

%List of Symbols
%
$a$ & distance & \si{\meter} \\
$P$ & power & \si{\watt} (\si{\joule\per\second}) \\
%%Símbolo, nome e unidade

\addlinespace % espaçamento para separar os símbolos romanos dos grego

$\omega$ & angular frequency & \si{\radian} \\

\end{symbols}



%----------------------------------------------------------------------------------------
%	ACRÓNIMOS
%----------------------------------------------------------------------------------------

%Use GLS
\glsresetall
\printglossary[title=\listacronymname,type=\acronymtype,style=long]


%----------------------------------------------------------------------------------------
%	ACABOU - BOM TRABALHO
%----------------------------------------------------------------------------------------

\mainmatter % Começar numeração da página com numéros árabes (1,2,3 ...)

